\documentclass{article}
\usepackage[utf8]{inputenc}
\usepackage{amsmath}
\usepackage{amssymb}
\title{Home assignment 1}
\author{Jonas Borgström, Mohamad Alshalabi,Peter Thuresson}
\date{September 2021}

\begin{document}

\maketitle

\section{Question 32 In Chapter 2.}

The answer for all sub questions involve applying portioning for the random vector X. 
The variance-covaraince matrix is listed below. 
\newline 
 

 $\Sigma_x$= $\left[
\begin{array}{cc|ccc}
4 & -1 & 1/2 & -1/2 & 0\\
-1 & 3 & 1 & -1 & 0\\ \hline
1/2 & 1 & 6 & 1 & -1\\ 
-1/2 & -1 & 1 & 4 & 0\\ 
0 & 0 & -1 & 0 & 2
\end{array}
\right]
$\newline
This matrix includes the partitioned variance-covariance matrix.\newline 
This matrix is the same matrix 
$\Sigma_x$ =$\left[
\begin{array}{cc}
\Sigma_{11} & \Sigma_{12} \\
\Sigma_{21} & \Sigma_{22}
\end{array}
\right]$
similarly, the mean vector is partitioned the same manners.\newline 
$\mu$ = $\left[
\begin{array}{c}
2\\
4\\ \hline
-1\\
3\\
0
\end{array}
\right]
$\newline
\subsection{(a-2-32)}
Find E($X^{[1]} $) = $\mu^ {[1] }$ = $\left[
\begin{array}{c}
2\\
4
\end{array}
\right]
$\newline
\subsection{(b-2-32)}
Find  E($A*X^{[1]} $) = $A*\mu^{[1]}$ = $\left[
\begin{array}{cc}
1 & -1 \\
1 & 1
\end{array}
\right]
$*$\left[
\begin{array}{c}
2\\
4
\end{array}
\right]
$ = $\left[
\begin{array}{c}
-2\\
6
\end{array}
\right]
$\newline
\subsection{(c-2-32)}
Find Cov($X^{[1]}$) = $\Sigma_{11}$ = $\left[ 
\begin{array}{cc}
    4 & -1 \\
    -1 & 3
\end{array}
\right]$\newline
\subsection{(d-2-32)}
Find Cov($A*X^{[1]}$) = $A'$ *  $\Sigma_{11}$ * $A$ = $\left[
\begin{array}{cc}
1 & 1 \\
-1 & 1
\end{array}
\right]
$*$\left[ 
\begin{array}{cc}
4 & -1 \\
-1 & 3
\end{array}
\right]$* $\left[
\begin{array}{cc}
1 & -1 \\
1 & 1
\end{array}
\right]
$ = $\left[ 
\begin{array}{cc}
9 & 1 \\
1 & 5
\end{array}
\right]$\newline
\subsection{(e-2-32)}
Find E($X^{[2]} $) = $\mu^ {[2] }$ = $\left[
\begin{array}{c}
-1\\
3\\
0
\end{array}
\right]
$\newline
\subsection{(f-2-32)}
Find  E($B*X^{[2]} $) = $B*\mu^{[2]}$ = $\left[
\begin{array}{ccc}
1 & 1 &1 \\
1 & 1 &-2
\end{array}
\right]
$*$\left[
\begin{array}{c}
-1\\
3\\
0
\end{array}
\right]
$ = $\left[
\begin{array}{c}
2\\
2
\end{array}
\right]
$\newline
\subsection{(g-2-32)}
Find Cov($X^{[2]}$) = $\Sigma_{22}$ = $\left[ 
\begin{array}{ccc}
6 & 1 & -1 \\
1 & 4 & 0 \\
-1& 0 & 2 
\end{array}
\right]$\newline
\subsection{(h-2-32)}
Find Cov($B*X^{[2]}$) = $B$ *  $\Sigma_{22}$ * $B'$ = $\left[
\begin{array}{ccc}
1 & 1 &1 \\
1 & 1 &-2 
\end{array}
\right]
$*$\left[ 
\begin{array}{ccc}
6 & 1 & -1 \\
1 & 4 & 0 \\
-1& 0 & 2 
\end{array}
\right]$* $\left[
\begin{array}{cc}
1 & 1 \\ 
1 &1 \\
1 &-2 
\end{array}
\right]
$ = $\left[ 
\begin{array}{cc}
12 & 9 \\
9 & 24
\end{array}
\right]$\newline
\subsection{(i-2-32)}
Find Cov($X^{[1]},X^{[2]}$) = $ \Sigma_{12}$ = $ \Sigma_{21}'$ = $\left[
\begin{array}{ccc}
1/2 & -1/2 & 0 \\
1 & -1 & 0 
\end{array}
\right]$ \newline  
\subsection{(j-2-32)}
Find Cov($AX^{[1]},BX^{[2]}$) =  $A$*$ \Sigma_{12}$*$B`$ = $\left[
\begin{array}{cc}
1 & -1 \\
1 & 1
\end{array}
\right]
$*$\left[
\begin{array}{ccc}
1/2 & -1/2 & 0 \\
1 & -1 & 0 
\end{array}
\right]$ * $\left[
\begin{array}{cc}
1 & 1 \\ 
1 &1 \\
1 &-2 
\end{array}
\right]
$ = 
$\left[
\begin{array}{cc}
0 & 0 \\
0 & 0
\end{array}
\right]
$\newline
\section{Question 18 in Chapter 3.}
\subsection{a-3-18}
The consumption of energy is modeled as follow \newline
$Y= X{1} + X{2}+ X{3} + X{4}$\newline
The mean vector of the is listed in the question\newline 
$\overline{X}= \left[
\begin{array}{c}
0.776 \\
0.508\\
0.438\\ 
0.161
\end{array}
\right]$\newline
The sample mean is calculated as follow\newline
$\overline{Y}= \left[ 
\begin{array}{cccc}
1&1&1&1 
\end{array}
\right] * \left[
\begin{array}{c}
0.776 \\
0.508\\
0.438\\ 
0.161
\end{array}
\right]
= 0.777 + 0.508 + 0.438 + 0.161 = 1.873 $\newline 
The sample Variance is calculated as follow \newline
$S{y} = \left[ 
\begin{array}{cccc}
1&1&1&1 
\end{array}
\right] * S * \left [
\begin{array}{c} 
1\\
1\\ 
1\\ 
1
\end{array}
\right]=\left[ 
\begin{array}{cccc} 
1.76 & 1.397 & 0.511 & 0. 245 
\end{array}
\right] *\left [
\begin{array}{c} 
1\\
1\\ 
1\\ 
1
\end{array}
\right]= 1.76 + 1.397 + 0.511 + 0.245 = 3.913.$ 
\subsection{b-3-18}
To determine the excess of the petroleum consumption over natural gas we need to reconstruct the model in such 
\newline 
$Z = X{1} - X{2}$
The sample mean of Z is calculated by replacing the vector used in question above by\newline
$ \left [
\begin{array}{c} 
1\\-1\\0\\0\end{array}\right]$ \newline 
$\overline{Z}=  \left [
\begin{array}{c} 
1\\-1\\0\\0\end{array}\right] * \left[
\begin{array}{c}
0.776 \\
0.508\\
0.438\\ 
0.161
\end{array}
\right] = 0.776 - 0.508 = 0.258 $ \newline
Using the same argument as the first part of the question we can calculate the sample variance of Z as follow.\newline
$S{Z} = \left[ 
\begin{array}{cccc}
1&-1&0&0 
\end{array}
\right] * S * \left [
\begin{array}{c} 
1\\
-1\\ 
0\\ 
0
\end{array}
\right]=\left[ 
\begin{array}{cccc} 
0.221 & 0.067 & 0.045 & 0. 029 
\end{array}
\right] *\left [
\begin{array}{c} 
1\\
-1\\ 
0\\ 
0
\end{array}
\right]= 0.221 - 0.067 + 0 + 0 = 0.154 .$\newline
To calculate the Variance-covariance for Z within Y .....
$Cov(z,y)=  \left[ 
\begin{array}{cccc}
1&-1&0&0 
\end{array}
\right] * S *  \left [
\begin{array}{c} 
1\\
1\\ 
1\\ 
1
\end{array}
\right] = \left[ 
\begin{array}{cccc} 
0.221 & 0.067 & 0.045 & 0. 029 
\end{array}
\right] *  \left [
\begin{array}{c} 
1\\
1\\ 
1\\ 
1
\end{array}
\right] = 0.221 + 0.067 + 0.045 + 0.029 = 0.362$
\section{Question 4.16}
Let $\boldsymbol{X}_{1}$, $\boldsymbol{X}_2$. $\boldsymbol{X}_3$ and $\boldsymbol{X}_4$ be independent $N_p(\mu, \Sigma)$ random vectors. \\ \\
\subsection{a}
Find the marginal distributions for each of the random vectors $$ V_1 = \frac{1}{4}\boldsymbol{X}_1 - \frac{1}{4}\boldsymbol{X}_2 + \frac{1}{4}\boldsymbol{X}_3 - \frac{1}{4}\boldsymbol{X}_4$$
and
$$ V_2 = \frac{1}{4}\boldsymbol{X}_1 + \frac{1}{4}\boldsymbol{X}_2 - \frac{1}{4}\boldsymbol{X}_3 - \frac{1}{4}\boldsymbol{X}_4$$
\\
Solution / Motivation: \\
Using Result 4.8 (page 165), for $\boldsymbol{V}_1$ we get $c_1 = c_3 = \frac{1}{4}$ and $c_2 = c_4 = -\frac{1}{4}$. Setting $\mu_j = \mu$ for $j = 1, \dots, 4$ we can use $\sum^{4}_{j=1}c_j\mu_j = 0$ and $\sum^{4}_{j=1}c_j^2\Sigma = \frac{1}{4}\Sigma$. This gives us $\boldsymbol{V}_1 = (0, \frac{1}{4}\Sigma)$. \\
For $\boldsymbol{V}_2$ we get $b_1 = b_2 = \frac{1}{4}$ and $b_3 = b_4 = -\frac{1}{4}$. We can then use $\sum^{4}_{j=1}b_j\mu_j = 0$ and $\sum^{4}_{j=1}b_j^2\Sigma \sim \frac{1}{4}\Sigma$. This gives us $\boldsymbol{V}_2 \sim (0, \frac{1}{4}\Sigma)$. \\
Answer: \\
$\boldsymbol{V}_1 \sim (0, \frac{1}{4}\Sigma)$ \\
$\boldsymbol{V}_2 \sim (0, \frac{1}{4}\Sigma)$
\subsection{b}
Find the joint density of the random vectors $\boldsymbol{V}_1$ and $\boldsymbol{V}_2$ defined in $\boldsymbol{a}$. \\ \\
Solution / Motivation: \\
Again, using Result 4.8 (page 165) we have that $\boldsymbol{V}_1$ and $\boldsymbol{V}_2$ are jointly multivariate normal and that $(\sum_{j=1}^{4} = b_{j}c_{j})\Sigma = (\frac{1}{4}*\frac{1}{4}+\frac{-1}{4}*\frac{1}{4}+\frac{1}{4}*\frac{-1}{4}+\frac{-1}{4}*\frac{-1}{4})\Sigma = 0$. \\
Thus $
    \begin{bmatrix}
    \boldsymbol{V}_1 \\
    \boldsymbol{V}_2
    \end{bmatrix}$ is distributed by $N_{2p} = (0,
    \begin{bmatrix}
    \frac{1}{4}\Sigma & 0 \\
    0 & \frac{1}{4}\Sigma
    \end{bmatrix})$. \\
    We can then use definition $4-4$ (page 150) which gives us following equation \\
    $f(v1, v2) = $
    \begin{equation*}
\begin{split}
f(v1, v2) & = \frac{1}{(2\pi)^{\frac{p}{2}}\begin{vmatrix}
\Sigma
\end{vmatrix}^{\frac{1}{2}}} * exp(-(\boldsymbol{x}-\boldsymbol{\mu})^{T}\Sigma^{-1}({\boldsymbol{x} - \boldsymbol{\mu})/2} \\
 & = \frac{1}{(2\pi)^{p}\begin{vmatrix}
\frac{1}{4}\Sigma
\end{vmatrix}} * exp(-\frac{1}{2}(\boldsymbol{x}-\boldsymbol{\mu})^{T}\Sigma^{-1}({\boldsymbol{x} - \boldsymbol{\mu})/2}
\end{split}
\end{equation*}
\end{document}
